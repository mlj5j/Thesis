\chapter{CMS Particle and Event Reconstruction}
After an event is chosen to be stored by the trigger system, the output from all of the sub-detectors is saved and recorded to disk as "RAW" data.  These data contain information about the response of each sub-detector, such as tracker hits and energy deposition in the calorimeters.  As was mentioned in Chapter 4, shown in Table \ref{table:subdetsignals} and Figure \ref{fig:cmsslicewhitecolourfrench291016}, the CMS was designed such that each type of particle resulting from the $pp$ collisions at the IP would leave a distinct signature in the sub-detectors.  This allows for the information to be reconstructed into lists of physics object candidates such as photons, electrons, muons, etc and quantities such as missing transverse momentum.  The particle flow (PF) algorithm performs this reconstruction by first building tracks and calorimeter clusters.  These two elements are the inputs to the reconstruction of the aforementioned physics object candidates using a "link" algorithm.

\section{Tracks}
A combinatorial track finder algorithm based on the Kalman filtering technique uses the hits in the silicon tracker to reconstruct tracks of charged particles \cite{Kalmantracking:1987fm}.  Each iteration of the algorithm is comprised of three steps:
\begin{itemize}
	\item Seed generation:  Find a seed consisting of two to three hits that is compatible with a track from a charged particle.
	\item Track finding: Use pattern recognition to identify any hits that are compatible with the trajectory implied by the seed generated in the first step.
	\item Track fitting: Determine the properties of the track, such as origin, trajectory, and transverse momentum by performing a global $\chi^2$ fit.
\end{itemize}

The first iteration uses stringent requirements on the seeds and the $\chi^2$ of the track fit to pick out isolated jets which have very high purity.  The hits associated with these high purity tracks are then removed to reduce the combinatorial complexity for subsequent iterations.  This allows successive iterations to identify less obvious tracks by progressively loosening criteria while the removal of previously associated hits mitigates the likelihood of fake tracks being built.  


\section{Calorimeter clusters}
Calorimeter clusters are constructed using energy deposition information from the calorimeters.  Clusters are formed by first identifying the seed cell (ECAL crystal or HCAL scintillating tile) that corresponds to the local maxima of an energy deposit that is above a given threshold.  Neighboring cells are then aggregated to grow topological clusters if their signals are above twice the standard deviation of the level of electronic noise.  

\section{Object identification}
At this point the tracks and calorimeter clusters are linked to form a PF block.  This linkage is done with an algorithm that quantifies the likelihood that a given track and cluster were results of the same particle.  As PF locks are identified as object candidates they are removed from the collection prior to each subsequent iteration until all tracks and clusters have been assigned to a PF object candidate.  The following sections will outline how each of these PF objects is identified.
\subsection{Muons}
Muons are the easiest particle to identify, so they are the first objects reconstructed in the CMS.  PF Muons are classified in three categories depending on how their tracks are reconstructed:
\begin{itemize}
	\item Tracker muons:  Tracks reconstructed from the inner tracker having $p_T>0.5$ GeV and $|\vec{p}|>2.5$ GeV that, when propagated to the muon system, match at least one hit in the muon chambers.
	\item Stand-alone muons:  Tracks reconstructed only using hits in the muon system.
	\item Global muons:  Stand-alone muons that coincide with a track from the inner tracker.
\end{itemize}
After a muon is reconstructed it is given an identification or ID based on observables such as the $\chi^2$ of the track fit, how many hits were recorded per track, or how well the tracker and stand-alone tracks matched.  These IDs represent different working points (loose, medium, and tight) which correspond to increasing purity but decreasing efficiency as you move from loose toward tight.  
\subsection{Electrons}

\subsection{Photons} 

\subsection{Jets}

\section{Missing transverse momentum}

